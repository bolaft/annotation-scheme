\subsection{Goals}

Our goal is to develop an annotation scheme and protocol for the classification of message fragments in problem/solution oriented online conversations. The scheme should facilitate the problem/solution oriented analysis of message text, which, in turn, can be used for a number of applications:

\begin{itemize}
	\item Problem/solution oriented information retrieval (e.g. finding relevant documentation for a problem stated in natural language)
	\item Solution finding (e.g. the proposal of previous solutions that worked for a similar problem)
	\item Cross-modal analysis (e.g. using data from different sources to improve the efficiency of other tasks)
	\item Modality/channel relevance estimation (e.g. assisting users in finding the best place to find help)
	\item Etc.
\end{itemize}

Such a scheme would need to be:

\begin{itemize}
	\item Multimodal: applicable to different online mediums
	\item Exhaustive: capable of covering the entire text of a message
	\item Unambiguous: annotation should be made simple by following a decision tree
	\item Problem/solution oriented: useful in modelizing problems and their proposed solutions
\end{itemize}

A further objective is to use such annotated data to produce a sample segmentation of our email corpus so that we can test the validity of the hypotheses made in \cite{hernandez2014exploiting}.