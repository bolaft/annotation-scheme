\subsection{Dialog acts}
\label{subsec:dialog_acts}

This taxonomy of dialog acts is based on their communicative functions and is founded on DIT++\cite{bunt2009dit++}. 

Each dialog act can be labelled with one function class for each of the four differents dimensions. The selected dimensions are Domain, Interaction Management, Social Management and Feedback. Some communicative functions, called general-purpose, are dimension-generic.

Some dialog acts are always dependant upon another dialog act because of the nature of their function, they bear a functionaly relational functions (e.g. and answer is functionaly dependant upon a question). This relation may be backward-looking (an answer) or forward-looking (a question). We indicate a these relations with the symbols [$\rightarrow$ X] and [X $\leftarrow$].

But some dialog acts may also have an optional and non-functional relation to another (e.g. an ``inform'' act may \textit{explain} a previous ``instruct'' act). These links are called rhetorical relactions and are listed in \ref{subsubsec:rhetorical_relations}.

\subsubsection{Communicative functions}

\textbf{Information Transfer Functions}
\vspace{0.1cm}

These classes of functions deal with the exchange of information between participants. They are general-purpose.

\begin{itemize}
	\item Information Seeking Functions
		\begin{itemize}
			\item Question [Answer $\leftarrow$]
				\begin{itemize}
					\item Set Question
					\item Propositional Question
						\begin{itemize}
							\item \sout{Check Question} \textit{(also removed children ``posi-check'' and ``nega-check'', too deeply nested)}
						\end{itemize}
					\item Choice Question
				\end{itemize}
		\end{itemize}
	\item Information Providing Functions
		\begin{itemize}
			\item Inform
				\begin{itemize}
					\item Agreement
					\item Disagreement
						\begin{itemize}
							\item Correction
						\end{itemize}
					\item Answer [$\rightarrow$ Question]
						\begin{itemize}
							\item Confirm
							\item Disconfirm
						\end{itemize}
				\end{itemize}
		\end{itemize}
\end{itemize}

\textbf{Action Discussion Functions}
\vspace{0.1cm}

Dialog acts bearing action discussion functions are about performing an action. They are general-purpose.

\begin{itemize}
	\item Commissives
		\begin{itemize}
			\item Offer
				\begin{itemize}
					\item Promise
					\item Threat
				\end{itemize}
			\item Address Request
				\begin{itemize}
					\item Accept Request
					\item Decline Request
				\end{itemize}
			\item Address Suggestion
				\begin{itemize}
					\item Accept Suggestion
					\item Decline Suggestion
				\end{itemize}
			\item Other
		\end{itemize}
	\item Directives
		\begin{itemize}
			\item Request
				\begin{itemize}
					\item Instruct
				\end{itemize}
			\item Address Offer
				\begin{itemize}
					\item Accept Offer
					\item Decline Offer
				\end{itemize}
			\item Suggestion
			\item Other directives
		\end{itemize}
\end{itemize}

\textbf{Domain-related Functions}
\vspace{0.1cm}

These function classes are used to label dialog acts that are specific to online written conversations. They are NOT meant to label dialog acts directly concerned with explaining and solving problems. They are specific to the "Domain" dimension. These classes may vary depending on the conversation's medium.

\begin{itemize}
	\item Emote
	\item Provide non-verbal
		\begin{itemize}
			\item Provide raw data
			\item Provide hypertext link
			\item Provide media
		\end{itemize}
	\item Request non-verbal
		\begin{itemize}
			\item Request raw data
			\item Request hypertext link
			\item Request media
		\end{itemize}
	\item Forward Text
	\item Include Text
		\begin{itemize}
			\item Include Text for Framing
			\item Include Text for Evaluation
		\end{itemize}
	\item Medium Specific
		\begin{itemize}
			\item IRC
				\begin{itemize}
					\item Bip
				\end{itemize}
			\item Forum
				\begin{itemize}
					\item Close thread
					\item Sticky thread
					\item Edition artifact
					\item ...
				\end{itemize}
			\item ...
		\end{itemize}
\end{itemize}

\textbf{Dialogue Control Functions}
\vspace{0.1cm}

The functions of dialog acts that are used to ensure a successful communication between participants. They are specific to the 

\begin{itemize}
	\item Feedback Functions
		\begin{itemize}
			\item \sout{Auto-Feedback Functions}
			\item \sout{Allo-Feedback Functions}
		\end{itemize}
	\item Interaction Management Functions
		\begin{itemize}
			\item \sout{Turn Management} \textit{(rendered void by asynchronous format and technical specificities)}
			\item \sout{Time Management} \textit{(idem)}
			\item \sout{Contact Management} \textit{(idem)}
			\item \sout{Own Communication Management} \textit{(idem, and can be compensated by general-purpose functions)}
			\item \sout{Partner Communication Management} \textit{(idem)}
			\item Discourse Structure Management
				\begin{itemize}
					\item Opening
					\item Preclosing
					\item Topic Introduction
					\item Topic Shift Announcement
						\begin{itemize}
							\item Topic Shift
						\end{itemize}
				\end{itemize}
		\end{itemize}
	\item Social Management Functions
		\begin{itemize}
			\item Salutation
				\begin{itemize}
					\item Initial Greeting
					\item Return Greeting
				\end{itemize}
			\item Self-introduction
				\begin{itemize}
					\item Initial self-introduction
					\item Return self-introduction
				\end{itemize}
			\item Apology
				\begin{itemize}
					\item Apology
					\item Apology-downplay
				\end{itemize}
			\item Gratitude Expression
				\begin{itemize}
					\item Thanks
					\item Thanks-downplay
				\end{itemize}
			\item Valediction
				\begin{itemize}
					\item Initial goodbye
					\item Return goodbye
				\end{itemize}
		\end{itemize}
\end{itemize}

\subsubsection{Communicative function qualifiers}
\label{subsubsec:communicative_function_qualifiers}

\begin{itemize}
	\item Modality/Certainty
	\item Conditionality
	\item Partiality
	\item Mode/Sentiment
\end{itemize}

\subsubsection{Object-oriented function qualifiers}
\label{subsubsec:communicative_function_qualifiers}

In online conversations bearing expressions of need and attempts at answering to those needs, we consider that sentences may bear information on two types of abstract objects : problems, and solutions.

The following qualifiers define the kind of information a sentence brings about either of these objects.

\begin{itemize}
	\item Problem
		\begin{itemize}
			\item Description
				\begin{itemize}
					\item Matter
					\item Symptoms
					\item Context
					\item Explanation
				\end{itemize}
			\item Action report
				\begin{itemize}
					\item Problem trigger(s)
					\item Attempted solutions
					\item Reproduction procedure
				\end{itemize}
			\item Goal
			\item User profile
			\item Resolution status
		\end{itemize}
	\item Solution
		\begin{itemize}
			\item Description 
				\begin{itemize}
					\item Resolution procedure
					\item Explanation
				\end{itemize}
			\item Constraints
			\item Limitations
			\item Validation status
		\end{itemize}
\end{itemize}

\subsubsection{Feedback relations}
\label{subsubsec:rhetorical_relations}

There are only two types of feedback relations:

\begin{itemize}
	\item Feedback
	\item Feedback elicitation
\end{itemize}

\textbf{Feedback relation qualifiers}

Two types of qualifiers: polarity, and feedback dimension.

Polarity:

\begin{itemize}
	\item Negative
	\item Positive
\end{itemize}

Feedback dimension:

\begin{itemize}
	\item Execution
	\item Evaluation
	\item Interpretation
	\item Perception
	\item Attention
\end{itemize}

\subsubsection{Rhetorical relations}
\label{subsubsec:rhetorical_relations}

\begin{itemize}
	\item Elaboration
	\item Motivation
	\item Justification
	\item Cause
	\item Exemplification
	\item Conclusion
	\item Summary
	\item ...
\end{itemize}