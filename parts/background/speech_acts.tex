\subsection{Speech Acts}

Speech act theory \cite{austin1975things} attempts to describe utterances in terms of communicative function (e.g. question, answer, thanks...). Indeed, utterances are not limited to their semantic content, they also have a communicative function : a goal, and an effect. For Austin, they can be analyzed at three levels: locutory (the linguistic characteristics of the utterance), illocutory (the intention of the speaker) and perlocutory (the real-world effects of the utterance). When we refer to speech acts\footnote{Also known as dialog acts in the context of interactive conversations}, we are interested in the illocutory level of utterance analysis. 

Thus, in most works, it is in terms of speech acts that interactions between participants of a conversation are modeled. Austin considers utterances as actions performed by a speaker ; this is based on the idea that every enunciation is the realization of a social act. Verbs that specify these actions are called \textit{performative verbs}, such as when someone says "I grant you the title of captain". But speech acts are not only constituted by these kinds of verbs. \cite{searle1976taxonomy} offers five classes of dialog acts : assertives (assertion...), directives (order, request, advice, etc.), commissives (promise, invitation, etc.), expressives (congratulations, thanks, etc.) and declarations (war declaration, nomination, baptism, etc.).

There are a number of different speech act taxonomies \cite{traum200020}. Two of them can be considered foundational: first Austin's and then Searle's. Both contain five classes of speech acts, that can be defined as such:

% \todo{Réécrire définitions (en l'état copié/collé Wikipédia)}

\begin{itemize}
	\item Assertives: speech acts that commit a speaker to the truth of the expressed proposition, e.g. reciting a creed.
	\item Directives: speech acts that are to cause the hearer to take a particular action, e.g. requests, commands and advice.
	\item Commissives: speech acts that commit a speaker to some future action, e.g. promises and oaths.
	\item Expressives: speech acts that express the speaker's attitudes and emotions towards the proposition, e.g. congratulations, excuses and thanks.
	\item Declaratives: speech acts that change the reality in accord with the proposition of the declaration, e.g. baptisms, pronouncing someone guilty or pronouncing someone husband and wife.
\end{itemize}

Table \ref{fig:fundamentalTaxonomies} provides examples of these two taxonomies. Table \ref{fig:emailTaxonomies} presents a few more recent taxonomies specifically used in the context of online conversation analysis.

\begin{table}
	\begin{tabularx}{\textwidth}{c c c}
		\toprule
		\cite{austin1975things} & \cite{searle1976taxonomy} & Examples \\
		\midrule
		Verdictives & Declarations & To condemn, decree... \\
		Exercitives & Directives & To command, order, forgive... \\
		Commissives & Commissives & To promise, guarantee, bet, swear... \\
		Behabitives & Expressives & To apologize, thank, criticize... \\
		Expositives & Assertives & To assert, deny, postulate... \\
		\bottomrule
	\end{tabularx}
	\caption{Foundational taxonomies for speech acts categorization}
	\label{fig:fundamentalTaxonomies}
\end{table}

\begin{table}
	\begin{tabularx}{\textwidth}{c c c}
		\toprule
		ActS & Corpus & Reference \\
		\midrule
		Disclosure &  & \\
		Edification &  & \\
		Advisement &  & \\
		Confirmation & Multi-domain & \cite{Lampert_classifyingspeech} \\
		Question &  & \\
		Acknowledgment &  & \\
		Interpretation &  & \\
		Reflection &  & \\
		\midrule
		Direct request &  & \\
		Question-request &  & \\
		Open question &  & \\
		First person commitment & Corporate email & \cite{de2013classification} \\
		First person expression of feeling &  & \\
		First person other &  & \\
		Other statements &  & \\
		\midrule
		Accept response &  & \\
		Acknowledge and appreciate &  & \\
		Action motivator &  & \\
		Polite mechanism &  & \\
		Rhetorical question &  & \\
		Open-ended question & BC3 & \cite{JanAAAI08} \\
		Or/or-clause question &  & \\
		Wh-question &  & \\
		Yes-no question &  & \\
		Reject response &  & \\
		Statement &  & \\
		Uncertain response &  & \\
		\bottomrule
	\end{tabularx}
	% \todo[inline]{À enrichir}
	\caption{Examples of speech act taxonomies specific to online conversation analysis}
	\label{fig:emailTaxonomies}
\end{table}